\section{Einleitung / \"Uberblick}
Die statische Analyse, auch Quellcode Analyse, ist ein statisches Software-Testverfahren. Sie ist ein erster und unverzichtbarer Schritt verschiedenster Software-Qualitäts-Kontrollen. Dabei wird ein Verständnis für die Code-Struktur hergestellt. Sie trägt dazu bei, den Code an definierte Standards anzupassen.

Die Analyse ermöglicht eine frühzeitige Erkennung und somit die Korrektur von Fehlern mit geringerem Aufwand. Durch das frühe Eingreifen wird verhindert, dass scheinbar kleine Fehler sich manifestieren und Wochen, Monate oder Jahre später zu erheblichen Problemen führen.

Bei der statischen Analyse wird -- im Vergleich zu dynamischen Testverfahren -- die zu testende Software nicht ausgeführt. In der Regel wird dabei mit dem Quelltext gearbeitet und lässt sich somit eher zu den White-Box-Testverfahren zuordnen, wie später noch erwähnt wird gibt es auch Verfahren die mit dem kompilierten Code arbeiten und eher dem Black-Box-Testverfahren zuzuordnen sind.

Die Analyse kann vollständig manuell durchgeführt werden; dies ist aber mit erheblichem Aufwand verbunden. Daher werden automatisierte Tools eingesetzt. Das erste Tool dieser Art war ``Lint'', das in den 1970er Jahren von den Bell Labs für die Programmiersprache C entwickelt wurde. Das heutzutage bekannteste Tool ist ``FindBugs'', auf dessen Funktionsweise und Umfang wir später noch näher eingehen. Im Vergleich zu den frühen automatisierten Hilfen wird heute neben funktionalen Fehlern auch sogenannter ``Code-Smell'' erkannt. Dies können beispielsweise duplizierter oder ``toter Code'' (nicht-verwendeter Code) sein. Dabei kommen Techniken wie die \emph{Taint-Analyse}\footnote{vgl. \url{http://en.wikipedia.org/wiki/Taint_checking}} oder die \emph{Datenflussanalyse}\footnote{vgl. \url{http://de.wikipedia.org/wiki/Datenflussanalyse}} zum Einsatz. 

Das Ziel der statischen Analyse ist momentan nur das Aufzeigen von möglichen Fehlern, es wird keine automatische Korrektur des Codes vorgenommen. Idealerweise würden solche Tools mit einem hohen Maß an Zuverlässigkeit mutmaßliche Fehler und Sicherheitslücken automatisch erkennen und beheben. Dies ist jedoch noch nicht realisierbar. Momentan gibt es noch immer eine hohe Rate von Fehleinschätzungen.

Nichtsdestotrotz ist die statische Analyse in der heutigen Programmierung nicht mehr wegzudenken und dient als hervorragende Hilfestellung für Programmierer und Tester um vor allem kritische Teile des Codes zu analysieren und Fehler effizient zu finden.
Die Tools selbst lassen sich in der Regel problemlos in gängige Entwicklungsumgebungen integrieren und können somit unmittelbar während der Entwicklung den Programmierer auf eventuelle Fehler hinweisen.

\paragraph{Beispiele für erkennbare Fehler:}
Folgende Liste markanter Fehler aus den Bereichen Sicherheit, Zuverlässigkeit, sowie Code-Standards und Wartbarkeit bieten einen guten Eindruck, wie wichtig die statische Analyse bei der Entwicklung ist.\footnote{vgl. \url{http://www.klocwork.com/products/insight/klocwork-truepath/index.php}, \url{http://www.hitex.com/index.php?id=3360&L=2}}
\begin{itemize}
\item Syntaktische Fehler (z.B. Tippfehler)
\item Stilfehler, Einhaltung von Standards
\item Software Metriken
\item ``Bad Smells''
\item Leere Anweisungen (z.B. Exceptions)
\item Speicherlecks
\item Zerstörung von Speicherinhalten
\item Zugriffe über NULL-Zeiger
\item Index außerhalb der Grenzen
\item Unsichere Zeigerarithmetik
\item Benutzung von Speicher nach Freigabe
\item Inkonsistente Freigabe
\item Konstruktor-/Dekonstruktor-Lecks
\item Fehlerhafte Verwendung von virtuellen Member-Funktionen
\item Arithmetische Fehler
\item Nicht initialisierte / unbenutzte Variable
\item Redundanter / überflüssiger Code
\item Division durch null
\item 32/64-bit Kompatibilitätsprobleme
\item Pufferüberläufe
\item Nicht-validierte Benutzereingaben
\item Mögliche Sicherheitsproblematiken (z.B. Injections, Cross-Site-Scripting)
\item Veraltete Methoden
\item Gelockte und nicht-gelockte Zugriffe auf dieselbe Variable per Threads
\item Endlosschleifen
\item Toter oder unerreichbarer Code
\end{itemize}

\paragraph{Stärken, Schwächen und Grenzen der statischen Analyse:}
Im folgenden werden die Stärken und Schwächen, als auch die Problematik von falsch positiv und falsch negativ Aussagen aufgezeigt.\footnote{vgl. \url{https://www.owasp.org/index.php?title=Static_Code_Analysis\&oldid=132351}}
\subparagraph{Stärken:}
\begin{itemize}
\item Skaliert sehr gut, da nur auf dem Quelltext bzw. Kompilat gearbeitet wird
\item Sehr hilfreich sowohl bei einfachen Fehlern (z.B. Tippfehler) als auch bei komplexeren Interaktionen (z.B. Pufferüberläufe, SQL-Injections, etc.)
\end{itemize}

\subparagraph{Schwächen:}
\begin{itemize}
\item Viele Sicherheitslücken sind sehr schwer automatisch zu finden, z.B. bei Authentifizierung, Zugriffskontrollen, unsichere Verwendung von Kryptographie, etc.
\item Hohe Falsch-Positiv-Rate
\item Konfigurationsprobleme können nicht bzw. nur schwer erkannt werden
\item Schwer nachzuweisen, dass ein gefundenes Sicherheitsproblem eine tatsächliche Gefährdung darstellt
\item Schwierigkeiten bei der Analyse von Codestücken, die nicht kompiliert werden können, z.B. wenn entsprechende Referenz-Bibliotheken nicht mitgeliefert wurden zum Testen
\end{itemize}

\subparagraph{Falsch Positiv:}
Eine statische Code-Analyse führt häufig zu falsch eingeschätzten positiven Ergebnissen, bei denen das Tool eine Schwachstelle aufzeigt, welche eigentlich keine ist. Das tritt auf, wenn das Tool die Integrität und Sicherheit des Datenflusses durch die Applikation von Eingang zum Ausgang nicht immer nachvollziehen kann. Beispiele dafür sind, wenn Programmcode analysiert wird, der mit Closed-Source-Software oder externen Komponenten interagiert.

\subparagraph{Falsch Negativ:}
Falsch negativ Ergebnisse können ebenso auftreten, wenn Schwachstellen auftreten, diese aber nicht vom Tool gemeldet werden. Dies kann dann vorkommen, wenn eine neue Schwachstelle in einer externen Komponente entdeckt wurde oder das Tool keine Kenntnis von der Laufzeitumgebung hat und ob diese sicher konfiguriert wurde.

\paragraph{Abgrenzung zu Code Reviews:}
Statische Analyse darf nicht verwechselt werden mit Code Reviews. Die Analyse kann Teil eines Review Prozesses sein und optimiert diesen insofern, dass durch die vorgeschlagenen Verbesserungen der Code syntaktisch korrekt und übersichtlich vorbereitet ist.

Ein Code Review geht prinzipiell noch einen Schritt weiter und verfolgt teilweise auch andere Ziele. So prüft eine statische Analyse beispielsweise, ob eine Funktion generell dokumentiert ist, bei einem Code Review wird zusätzlich betrachtet, ob die Funktion ausreichend und verständlich erklärt wurde.

Somit ist ein Code Review immer eine manuelle Prüfung des Quellcodes durch mindestens eine weitere Person. Dabei wird die Aufmerksamkeit auf Punkte wie die Verständlichkeit des Codes oder sinnvoll benannte Variablen gelegt. Ebenso können Verbesserungen im Code bzw. in dessen Struktur und mögliche alternative Lösungsansätze betrachtet werden.
